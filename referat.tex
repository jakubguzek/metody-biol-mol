\documentclass[two column, twoside, a4paper]{article}

\usepackage[utf8]{inputenc}
\usepackage{dblfloatfix}
\usepackage{float}
\usepackage[polish]{babel}
\usepackage[T1]{fontenc}
\usepackage[backend=biber, maxbibnames=3, style=nature, autocite=inline]{biblatex}
\usepackage{fancyhdr}
\usepackage{titlesec}
\usepackage{blindtext}
\usepackage{cuted}
\usepackage[most]{tcolorbox}
\usepackage[columnsep = 1cm,
	    lmargin = 0.6in,
	    rmargin = 0.4in,
	    tmargin = 0.5in,
	    bmargin = 0.65in,
	    headsep = \baselineskip]{geometry}

\addbibresource{$BIB}

\title{Metody biologii molekularnej, w badaniach wirusów}
\author{Jakub J. Guzek}
\date{}

% Section Formatting
\titleformat{\section}
{\sc \bfseries \Large}
{}
{0em}
{}[\titlerule]

\titleformat{\subsection}
{\bfseries \large}
{}
{0em}
{}

\titleformat{\subsubsection}
{\bfseries}
{}
{0em}
{}

% Box formatting
\tcbset{enhanced, colback=orange!15!white, sharp corners, boxrule = 0pt, frame hidden}

\pagestyle{fancy}
\fancyhf{}
\fancyhead[RE, LO]{Szkoła Główna Gospodarstwa Wiejskiego}
\fancyhead[LE, RO]{Biotechnologia}
\fancyfoot[RE, LO]{Jakub J. Guzek}
\fancyfoot[LE, RO]{\thepage}
\fancyfoot[CE,CO]{Metody biologii molekularnej, w badaniach wirusów}
\renewcommand{\footrulewidth}{0.05pt}

\begin{document}

\begin{strip}
{\sc \bfseries \huge \fontfamily{phv}\selectfont Metody biologii molekularnej w badaniach

\vspace{3pt} wirusologicznych} \vspace{\baselineskip}

{\bfseries \Large Jakub J. Guzek}

{Szkoła Główna Gospodarstwa Wiejskiego, Biotechnologia, Nr. albumu: 195528}\vspace{\baselineskip}

\hrule
\end{strip}

\section{Wstęp}

Biologia molekularna jest jedną z najszybciej rozwijających się dziedzin wśród nauk biologicznych. Jej metody umożliwiają w dniu dzisiejszym sekwencjonowanie całych genomów -- co kiedyś było niemożliwe, lub pochłaniało ogromne zasoby zarówno pieniężne jak i czasowe -- w stosunkowo niedługim czasie. W rozwoju tych metod pomogły badania prowadzone na bardzo wielu, różnych organizmach i wirusach. Największy postęp przyniosły ogromne projekty takie jak \textbf{Projekt Poznania Genomu Człowieka} (\textit{Human Genome Project}) \autocite{IHGSC2001}, poznanie genomu neandertalczyka \autocite{Prufer2014}, czy projekt poznania genomu jęczmienia \autocite{IBGSC2012}.

\begin{figure}[h]
\begin{tcolorbox}
	\centering
	\includegraphics[width=\textwidth]{./sequenced_genomes.pdf}
	\caption{Ostatnie lata odnotowują bardzo szybki wzrost sekwencjonowanych genomów. Sekwencjonowanie genomów wirusowych odnotowuje jednak o wiele niższy wzrost niż prokariotycznych i eukariotycznych. Wykres przedstawia ilość genomów dodanych do bacy NCBI \autocite{NCBI} w danym roku.}\label{fig::seq_trends}

	\footnotetext{Autor pracy jest autorem wykresu, opracowanego na podstawie internetowej bazy danych NCBI. Kod wykresu można znaleźć pod adresem \textit{https://github.com/jakubguzek/metody-biol-mol}}
\end{tcolorbox}
\end{figure}

Jak wspomniano wyżej, dzisiejsze metody umożliwiają sekwencjonowanie całych genomów w stosunkowo niedługim czasie, a wraz z opracowywaniem metod sekwencjonowania trzeciej i czwartej generacji możliwe będzie sekwencjonowanie ich w czasie rzeczywistym \autocite{Brown2019}. Prócz tego mikromacierze, hybrydyzacja i chromatografia powinowactwa umożliwiają badanie transkryptów genów i tworzenie transkryptomów (zbiorów wszystkich cząsteczek RNA w komórce) oraz zestawu białek komórkowych zwanego proteomem. Daje to możliwości badania komórki i interakcji molekularnych jakie są w niej obecne i umożliwia rozwój takim dziedzinom jak biologia systemów i interaktomika. Temu wszystkiemu towarzyszy oczywiście doskonalenia narzędzi informatycznych używanych do badań z zakresu \textit{-omics}, którym zajmuje się bioinformatyka.

Które z tych metod są wykorzystywane w badaniu wirusów? Wiele z nich zostało opracowanych przy pracy na wirusach, zwłaszcza w początkowych dniach biologii molekularnej, gdy sekwencjonowanie genomów było przedsięwzięciem wielokrotnie trudniejszym niż dzisiaj, gdyż genomy wirusowe są niedużych rozmiarów, były więc atrakcyjnym materiałem badawczym. Wraz z postępem w tej dziedzinie wzrosła jednak ilość sekwencjonowanych genomów prokariotycznych i eukariotycznych (Rysunek \ref{fig::seq_trends}). Na dzień dzisiejszy w bazie danych NCBI jest ponad 30 tys. rekordów dla genomów wirusowych i ponad 200 tys. dla genomów prokariotycznych.

\section{Sekwencjonowanie genomu}

Celem badań genomicznych jest poznanie sekwencji genomu badanego organizmu lub wirusa, Techniki używane w czasie projektów poznawania genomów noszą nazwę sekwencjonowania i polegają zwykle na sekwencjonowaniu krótkich fragmentów wyizolowanych z badanych próbek. Długości takich sekwencji waha się w zależności od używanej metody od 750 bp. do 700 Mb., a w wypadku metod trzeciej i czwartej generacji wirtualnie nie ma ograniczenia gdyż zachodzi w czasie rzeczywistym \autocite{Brown2019}.

Pierwsze metody sekwencjonowania genomu zostały opracowane w latach 70 XXw. \textbf{Metoda terminacji łańcucha} opracowana przez F. Sangera \autocite{Sanger1977}, szybko zyskała popularność i mimo wielu ograniczeń jest stosowana do dzisiaj w projektach sekwencjonowania niedużych genomów \autocite{Brown2000} i mniejszych fragmentów. Drugą metodą opracowaną w tej samej dekadzie było \textbf{sekwencjonowanie metodą degradacji chemicznej}, które nie zyskało takiej popularności jak metoda terminacji łańcucha ze względu na wykorzystywanie toksycznych odczynników i mniejsze możliwości automatyzacji.

Techniki opracowane w kolejnych latach, umożliwiające sekwencjonowanie wielu tysięcy fragmentów jednocześnie nazywamy \textbf{technikami sekwencjonowania nowej generacji}.

\subsection{Sekwencjownoanie pierwszej generacji}

Jak już wspomniano metoda terminacji łańcucha opracowana przez F. Sangera, w latach 70, zyskała na popularności i została wdrożona w wielu dużych projektach takich jak Projekt Poznania Genomu Człowieka \autocite{IHGSC2001}. Oparta jest ona na zdolności jednoniciowego DNA, różniącego się długością do rozdziału na żelu poliakryloamidowym poprzez elektroforezę.

Elektroforezę taką można przeprowadzić w kapilarze o długości 50-80cm i średnicy $\sim 0,1mm$, co umożliwia rozdzielenie cząsteczek jednoniciowego DNA o długości do 1500 bp.

Samo sekwencjonowanie przeprowadza się przy użyciu polimerazy DNA, o pożądanych cechach takich jak:
\begin{itemize}
\item duża procesywność
\item nieznaczna aktywność egzonukleazy $5'\rightarrow3'$ lub jej brak
\item nieznaczna aktywność egzonukleazy $3'\rightarrow5'$ lub jej brak
\end{itemize}

W pierwszym etapie przyłączeniu do sekwencjonowanych fragmentów ulegają oligonukleotydy, pełniące rolę starterów dla syntezy nowej, komplementarnej nici DNA. W tej reakcji oprócz czterech, typowych trifosforanów deoksyrybonukleotydów (dATP. dCTP, dGTP, dTTP) jako substraty biorą udział także trifosforany dideoksyrybonukleotydów (ddATP, ddCTP, ddGTP, ddTTP), wyznakowane charakterystycznymi dla siebie znacznikami fluorescencyjnymi.

Polimeraza DNA przyłącza dideoksyrybonukleotydy tak samo jak przyłączyłaby normalne deoksyrybonukleotydy, jednak przyłączenie ddNTP uniemożliwia dalsze wydłużanie łańcucha i prowadzi do terminacji syntezy, na skutek braku grupy 3' hydroksylowej. ddNTPs są w środowisku reakcji obecne w stężeniu niższym niż dNTPs, terminacja nici zachodzi więc w różnych odległościach od startera, powodując powstanie wielu nici o różnych długościach, z których każda ma na końcu terminalnym dideoksyrybonukleotyd.

\begin{figure}[h]
\begin{tcolorbox}
	\centering
	\includegraphics[width=\textwidth]{./figury/sekwencja_metoda_terminacji.png}
	\caption{Odczytywanie sekwencji uzyskanej podczas sekwencjonowania metodą terminacji łańcucha (A) Każdy dideoksyrybonukleotyd jest wyznakowany innym fluorochromem. Detektor fluorescencji wykrywa jaki ddNTP znajduje się w przesuwającym się przed nim prążku i przesyła informacje do systemu obrazującego przetworzone dane. (B) Wydruk sekwencjonowania DNA (Na podstawie Brown 2019)}\label{fig::seq_chain_term}
\end{tcolorbox}
\end{figure}

Aby ustalić sekwencje badanego fragmentu należy rozdzielić uzyskane w tej procedurze fragmenty na żelu poliakryloamidowym i ustalić która cząsteczka kończy się jakim ddNTP (Rysunek \ref{fig::seq_chain_term}). Robi się to przy pomocy detektora fluorescencji, który rozróżnia znaczniki dołączone do poszczególnych dideoksyrybonukleotydów.

Zastosowanie techniki terminacji łańcucha umożliwia sekwencjonowanie fragmentów o długości do 750 bp. Automatyczne sekwenatory z licznymi kapilarami mogą jednak sekwencjonować wiele fragmentów równolegle co znacznie zwiększa możliwości tej metody \autocite{Brown2019}

Pierwszymi zsekwencjonowanymi genomami były genomy Bakteriofaga MS2 (ssRNA) i Bakteriofaga $\Phi X174$ \autocite{Sanger1978} (DNA), czyli genomy wirusowe. Genomy wirusów są relatywnie łatwe do zsekwencjonowania ze względu na ich (zwykle) nieduże rozmiary. Genom Bakteriofaga $\Phi X714$ został zsekwencjonowany metodą terminacji łańcucha.

\subsection{Sekwencjonowanie nowej generacji}

Metody nowej generacji pozwalają uzyskać dużą ilość danych, w krótkim czasie dzięki czemu umożliwiają złożenie nawet dużych genomów w znacznie krótszym czasie niż metoda terminacji łańcucha. Niosą one za sobą zwykle wyższe wymagania obliczeniowe, jednak w dzisiejszych czasach nie jest to dużym problemem, gdyż moc obliczeniowa konieczna do analizy tych danych jest względnie tania.

Istnieje wiele metod nowej generacji jednak ich wspólną cechą jest etap poprzedzający właściwe sekwencjonowanie, polegający na przygotowaniu biblioteki fragmentów DNA. Konstruowanie takiej biblioteki polega na unieruchamianiu fragmentów na podłożu stałym w formacie macierzy masowo równoległej. Umożliwia to przeprowadzenie wielu reakcji sekwencjonowania jednocześnie. Fragmenty na takiej macierzy mają zwykle niewielką długość (do 500 bp.). Unieruchamianie fragmentów przeprowadza się zwykle jednym z dwóch sposobów:
\begin{itemize}
	\item Ligacja końców fragmentów z adaptorami, czyli krótkimi dwuniciowymi kawałkami DNA, których sekwencje są komplementarne do oligonukleotywów związanych z mikromacierzą.
	\item Wykorzystanie podłoża w postaci małych metalicznych kulek pokrytych białkiem streptawidyną. W tym wypadku fragmenty DNA ligowane są z adaptorami, które mają przyłączone na końcach 5' znaczniki w postaci biotyny, która może tworzyć silne wiązania ze streptowidyną.
\end{itemize}
Następnie unieruchomione fragmenty amplifikuje się wykorzystując PCR, co umożliwia wytworzenie wystarczającej ilości kopii do zsekwencjonowania \autocite{Godwin2016}\autocite{Brown2019}.

Dalsze kroki różnią się znacząco w różnych metodach.

\begin{figure*}[bp]
	\begin{tcolorbox}
		\centering
		\includegraphics[width=\textwidth]{./figury/illumina.png}
		\caption{\textbf{Sekwencjonowanie metodą cyklicznej odwracalnej terminacji}. Po amplifikacji fragmentów na macierzy, mieszanina starterów, i zmodyfikowanych nukleotydów jest dodana na macierz, wraz z polimerazą DNA. Każdy nukleotyd ma grupę blokującą przy 3' węglu deoksyrybozy (tutaj grupa 3'-O-metyloazydkowa) i jest oznaczony charakterystycznym dla siebie fluorochromem (F),  W czasie każdego cyklu fragmenty ulegną wydłużeniu o jeden nukleotyd. Wówczas detektor mierzy fluorescencję z każdego miejsca na macierzy, Na koniec fluorochrom i grupa blokująca są wycinane \autocite{Godwin2016}.}\label{fig::illumina}
	\end{tcolorbox}
\end{figure*}

\textbf{Sekwencjonowanie metodą odwracalnej terminacji}, nazywana też sekwencjonowaniem w technologii Illumina (Rysunek \ref{fig::illumina}) wykorzystuje zmodyfikowane nukleotydy, które po dołączeniu przez polimerazę wydłużanego łańcucha powodują terminację. Metoda ta tym różni się jednak od metody F. Sangera, że tutaj terminacja jest odwracalna. Zaraz po odczytaniu na jakim nukleotydzie nastąpiła terminacja można z niego usunąć grupę blokującą, dołączoną do węgla 3' deoksyrybozy. Grupą blokującą może być znacznik fluorescencyjny, charakterystyczny dla danego nukleotydu. Mieszanina reakcyjna zawiera w wybranym momencie tylko zmodyfikowane nukleotydy, każdy krok syntezy powoduje więc terminację łańcucha. W każdym kroku detektor wykrywa znacznik fluorescencyjny i identyfikuje terminalny nukleotyd, Grupa blokująca jest następnie usuwana enzymatycznie. Metoda ta generuje odczyty o długości sekwencji do 300 bp. jednak dzięki wykorzystaniu macierzy masowo równoległej można uzyskać sekwencje o łącznej długości $20\,000$ Mb. \autocite{Godwin2016}.

\textbf{Sekwencjonowanie 454 (Roche)}, zwane także pirosekwencjonowaniem polega na wykrywaniu błysków chemoiluminescencji, wytwarzanych przez adenylilotransferazę siarczanową, rozkładającą cząsteczki pirofosforanu ($\mathrm{PP_{i}}$), który uwalniany jest przy każdym dołączeniu dNTP do syntezowanej nici. Każdy deoksyrybonukleotyd jest dodawany oddzielnie, jeden po drugim, a wzór chemoiluminescjencji jest wykorzystywany do ustalenia kolejności dodawania nukleotydów do nici przez polimerazę DNA \autocite{Brown2019} \autocite{Godwin2016}.

\textbf{Sekwencjonowanie ion torrent} wykorzystuje podobne podejście do sekwencjonowanie 454, system detekcji wykrywa jednak jony wodorowe, które są uwalniane wraz z $\mathrm{PP_{I}}$, zamiast samego pirofosforanu. Uwalnianie jony są wykrywane przez wrażliwy na jony tranzystor polowy (\textbf{ISFET} -- \textit{ion-sensitive field effect transistor}) \autocite{Brown2019}.

\textbf{Sekwencjonowanie przez ligację i wykrywanie oligonukleotydów} (\textbf{SOLiD} -- \textit{sequencing by oligonucleotide ligations and detection}). Metoda ta korzysta z innego podejścia niż metody nowej generacji opisane do tej pory. Sekwencja generowana jest na podstawie

\subsection{Sekwencjonowanie trzeciej i czwartej\\ generacji}

\blindtext[3]
\section{Anotacja genomu i badanie transkryptomu}

\blindtext[3]
\subsection{Mikromacierze dachówkowe}

\blindtext[3]
\subsection{RNA-Seq}

\blindtext[3]
\section{Badanie proteomu}

\blindtext[3]
\subsection{Metody oczyszczania białek}

\blindtext[3]
\subsection{Metody badania funkcji białek}

\blindtext[3]
\subsection{Interakcje między białkami}

\blindtext[3]
\section{Znaczenie dla wirusologii}

\blindtext[3]
\section{Podsumowanie}

\blindtext[3]
<++>

\printbibliography

\end{document}
